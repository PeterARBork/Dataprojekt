\documentclass{article}
\usepackage[utf8]{inputenc}
\usepackage{mathtools} %loads \usepackage{amsmath}
\usepackage{bm}
\usepackage{fullpage}
\usepackage{siunitx}
\usepackage{graphicx} % Required for inserting images
\usepackage{caption}
\usepackage{subcaption}
\usepackage{hyperref}
\hypersetup{
    colorlinks=true,
    citecolor=blue,
    linkcolor=blue,
    urlcolor=blue,
}

\newcommand{\df}[2]{\frac{\mathrm d #1}{\mathrm d #2}}
\newcommand{\pf}[2]{\frac{\partial #1}{\partial #2}}

\title{Random walk by hard sphere collisions with Gillespie inspired simulation}
\author{Peter A.~R.\ Bork}
\date{\today}

\begin{document}

\maketitle

%\tableofcontents
%\clearpage

\begin{abstract}
How will the position and velocity of a one-dimensional particle exposed to elastic collisions evolve in time? This text analyses a particle exposed to collisions with other particles of normally distributed velocities. Time between collisions is exponentially distributed  and depends on the particle speed. The first and second moments are derived analytically and compared to numerical simulations. The results show that hard sphere collisions result in normally distributed velocities and square positions, where the latter evolves in proportion to the square root of time, just like we expect for diffusion processes.
\end{abstract}

\section{The velocity of a colliding hard sphere}
In elastic collisions of hard spheres, the velocities of the particles after the collision can be calculated from their masses and previous velocities. Our focus will be on a heavy particle of mass $m_A$ with velocity $v_n$ before and $v_{n+1}$ after collision with a particle of lower mass $m_B$ and velocity $u_n$.
\begin{align}\label{eq:elasticcollision}
    v_{n+1} &= a v_n + b u_n \\
\end{align}
where we use
\begin{align}
    a &= \frac{m_A - m_B}{m_B + m_A} < 1\\
    b &= \frac{2m_B}{m_A + m_B} \ll 1
\end{align}
The rule above suggests a difference equation for the evolution of particle velocity. We will take the other particle velocities to be normally distributed,
\begin{align}
    u_n \sim \mathcal N(0, \sigma^2)
\end{align}
For now, the variance $\sigma^2$ is left undetermined (but its $k_B T)$.

We can learn the evolution of expected velocities using induction starting from an initial condition $v_0$.
\begin{align}
    \langle v_1 \rangle &= \langle a v_0 + b u_0 \rangle = a v_0 \\
    \langle v_2 \rangle &= \langle a v_1 + b u_1 \rangle = a^2 v_0 \\
    \vdots \\
    \langle v_n \rangle &= a^n v_0 \longrightarrow 0 \quad\text{for } n \longrightarrow \infty
\end{align}
In other words, any initial velocity will dissipate and the expected velocity will become zero if given enough time. But what of the expected square velocity? Using again induction
\begin{align}
    \langle v_1^2 \rangle &= \langle (a v_0 + b u_0)^2\rangle 
    = \langle a^2 v_0^2 + 2ab v_0 u_0 + b^2 u_0^2 \rangle
    = a^2 \langle v_0^2 \rangle + b^2 \sigma^2
    \\
    \langle v_2^2 \rangle &= a^2 \langle v_1^2 \rangle + b^2 \sigma^2
    = a^2(a^2 \langle v_0^2 \rangle + b^2 \sigma^2) + b^2 \sigma^2
    \\
    \vdots \\
    \langle v_n^2 \rangle &= a^{2n}v_0 + b^2 \sigma^2 \sum_{i=0}^n a^{2n}
    = a^{2n} v_0 + b^2 \sigma^2 \frac{1 - a^{2(n+1)}}{1 - a^2}
    \longrightarrow \frac{b^2 \sigma^2}{1-a^2} \quad \text{for } n \longrightarrow \infty
\end{align}
Here we used the convergence of the geometric series (with $a<1)$. This shows again that any initial velocity will dissipate but towards a constant fluctuation with variance $b^2 \sigma^2 / (1 - a^2)$.

\section{The evolution of time}
We take the collision frequency to be some $\lambda$ - the literature suggests formulas for gasses and for diluted solutions. But instead of reading up on that, I guessed it could be something like a mean free path $l$ divided by a 'relative velocity'
\begin{align}
    \lambda = \frac{|\bar v - v|}{l}
\end{align}
The time between two collisions will perhaps be exponentially distributed,
\begin{align}
    \Delta t_i \sim \operatorname{Exp}(\lambda) \\
    \langle \Delta t_i \rangle = \lambda^{-1}
\end{align}
In practice, we can compute it with
\begin{align}\label{eq:Deltat}
    \Delta t_i = -\lambda^{-1}\ln u^*_i \\
    u_i^* \sim \mathcal U(0, 1)
\end{align}
where $\mathcal U(0, 1)$ is the uniform distribution on the interval $[0, 1]$. Time then progresses such that
\begin{align}\label{eq:timeintegral}
    t_n = \sum_{i=0}^n \Delta t_i
\end{align}

\section{The particle trajectory}
A particle starting at $x_0$ will follow a trajectory given by the rule
\begin{align}\label{eq:Deltax}
    x_{n+1} = x_n + v_n \Delta t_n
\end{align}
Since the velocity and the collision-free time are take to be independent here, we have
\begin{align}
    \langle x_{n+1} \rangle = \langle x_n \rangle + \lambda^{-1} \langle v_n \rangle
    \longrightarrow \langle x_n \rangle \quad \text{for } n \longrightarrow \infty
\end{align}
In other words, when any initial velocity dies down, the particle has a constant expected position.

The expected square position is 
\begin{align}
    \langle x_{n+1}^2 \rangle &= \langle (x_n + v_n \Delta t )^2 \rangle
    = \langle x_n^2 + 2 v_n \Delta t x_n + v_n^2 \Delta t^2 \rangle
    = \langle x_n^2 \rangle + \langle v_n^2 \rangle \langle \Delta t^2 \rangle \\
    &= \langle x_n^2 \rangle + \left(a^{2n} v_0 + b^2 \sigma^2 \frac{1 - a^{2(n+1)}}{1 - a^2}\right) \frac{2}{\lambda^2}
\end{align}
We can simplify to the case where $v_0 = 0$ and find another geometric series
\begin{align}
    \langle x_{n+1}^2 \rangle - \langle x_n^2 \rangle
    = 2 \left(\frac{b \sigma}{\lambda}\right)^2 \frac{1 - a^{2(n+1)}}{1 - a^2}
    = k \left(1 - a^{2(n+1)} \right)
\end{align}
where we abbreviate with
\begin{align}
    k = 2 \left(\frac{b \sigma}{\lambda}\right)^2
\end{align}
The evolution of the expected square position is then
\begin{align}
    \langle x_n^2 \rangle
    &= k \left(n - \sum_{i=0}^n a^{2(n+1)} \right)
    = k \left(n - a^2 \sum_{i=0}^n a^{2n} \right)
    = k \left(n - a^2 \frac{1 - a^{2(n+1)}}{1-a^2} \right) \\
    \langle x_n^2 \rangle &\longrightarrow k \left( n- \frac{a^2}{1 - a^2} \right) \approx k n \quad \text{for } n \longrightarrow \infty
\end{align}
The particle will expand its reach with time, growing the variance with approximately $2 ( b \sigma / \lambda)^2$ at every collision.

\section{Numerical solution}
We can simulate the equations \eqref{eq:elasticcollision}, \eqref{eq:timeintegral}, \eqref{eq:timeintegral}, and \eqref{eq:Deltax} on the computer and compare our predictions for the statistics. This is done in figure~\ref{fig:Gillespiewalk}, where the statistics on simulations are in dark blue and the predicted statistics are light blue. The analysis and numerical implementation agree on the statistics.

\begin{figure}
    \centering
    \includegraphics[width=0.7\linewidth]{results/HardsphereBrownian.png}
    \caption{Hard sphere random walk. Grey lines are individual numerical simulations, blue lines are descriptive statistics of those simulations, and orange lines are analytical predictions. The plot on the left starts from $x_0 = 1000$, $v_0 = 0$ and runs $N=1000$ collisions. The plot on the right starts from $x_0=1000$ but non-zero $v_0 = 200$ and runs for 10 collisions. The other parameters were $a=0.5$, $b=0.01$, $l=10$, $\bar v = 100$, $\sigma = 1$, for $N=1000$ collisions in 100 parallel simulations.}
    \label{fig:Gillespiewalk}
\end{figure}

%\bibliographystyle{abbrv}
%\bibliography{kilder}

\end{document}
