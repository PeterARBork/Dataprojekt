\documentclass[10pt,reqno, usenames]{article}
\usepackage[T1]{fontenc}
\usepackage[utf8]{inputenc}
\usepackage[a4paper, hmargin={2.8cm, 2.8cm}, vmargin={2.5cm, 2.5cm}]{geometry}
\usepackage{hyperref}
\hypersetup{
    colorlinks=true,
    linkcolor=blue,
    filecolor=magenta,      
    urlcolor=blue,
}
\usepackage{amsmath,amssymb,amsthm} %Matematik
\usepackage{enumitem} %Lister
\usepackage[babel,farve, da]{ku-forside} % KU-forside

\newtheorem{saetning}{Sætning}
\newtheorem{lemma}[saetning]{Lemma}
\newtheorem{korollar}[saetning]{Korollar}
\newtheorem{definition}[saetning]{Definition}

\titel{Projekt n} %Skriv jeres projektnummer (1-8}
\undertitel{Projektnavn} %Skriv titlen på jeres projekt
\opgave{Dataprojekt} 
\forfatter{Gruppe 1A}%Skriv jeres gruppenummer
\dato{Institut for Matematiske Fag. \today} %"today" skriver automatisk dagens dato. 
\vejleder{Vejleder: Navn} %Skriv jeres vejleders navn

\begin{document}

\maketitle
\newpage

\section*{Gruppemedlemmer}
\begin{center}
\begin{tabular}{ p{5cm} |p{1.5cm} }
 \centering Navn &   KUid \\ \hline
 Forfatter 1 & aaa111  \\ \hline 
 Forfatter 2 & aaa111  \\ \hline 
 Forfatter 3 & aaa111  \\ \hline 
 Forfatter 4 & aaa111  \\ \hline 
 Forfatter 5 & aaa111  \\ \hline 
 Forfatter 6 & aaa111  \\ \hline 
\end{tabular}
\end{center}

\section*{Abstract}
Her skriver I et kort resumé på både dansk og engelsk af jeres projekt. Hvert resumé skal ikke fylde mere end en halv side. 

\newpage

\tableofcontents

\newpage

\section{Introduktion}

Nu kan i begynde jeres rapport. 
Når I undervejs har brugt en kilde, skal I skrive kildeinformationen ind i filen kilder.bib og når I vil henvise til den i teksten,  skal I bruge "cite" \cite{bog1}. (Se LATEX-kode). Hvis I ønsker at få en kilde med i litteraturlisten uden at henvise til den i teksten, kan I bruge "nocite" \nocite{hjemmseide1}. (Se LATEX-kode).  Hvis I vil læse mere om forskellige .bib formater, kan I kigge her: \href{https://verbosus.com/bibtex-style-examples.html}{https://verbosus.com/bibtex-style-examples.html}

\begin{figure}
    \centering
    \includegraphics[width=0.5\linewidth]{}
    \caption{Caption}
    \label{fig:placeholder}
\end{figure}


\section{Appendix}
Her er appendix. Husk at appendix ikke indgår i bedømmelsen af rapporten, men kan  bruges  til  
supplerende  udregninger,  figurer  eller  programkode, som  ikke  er nødvendig for forståelsen 
af rapportens indhold.





\bibliographystyle{abbrv}
\bibliography{kilder}

\end{document}
